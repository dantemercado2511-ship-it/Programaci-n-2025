\section{Ordenamiento}
Dependiendo del tipo del tiempo de ejecución, los algoritmos de ordenamiento 
pueden ser categorizados como rápidos o lentos. Serán rápidos todos aquellos que 
tengan un tiempo de ejecución menor o igual a $O(n \log n)$. Al resto se los considera 
lentos. Siguiendo esta definición se tiene:\\

\textbf{Lentos:}
\begin{enumerate}
    \item \textbf{\underline{Bubble sort}}
    \item \textbf{\underline{Selection sort}}
    \item Insertion sort
\end{enumerate}

\textbf{Rápidos:}
\begin{enumerate}
    \item Quicksort
    \item \textbf{\underline{Merge sort}}
    \item Heapsort
    \item \textit{Timsort}
\end{enumerate}

Aquellos en negrita y subrayados se han visto en la cátedra, los restantes no. 
\textit{Timsort} ha sido identificado en itálica porque es el método implementado 
por Python mediante el built--in \texttt{sort()}.

\begin{enumerate}[label=\textbf{\alph*)}]
    \item
    Investigue el algoritmo insertion sort y qué lo diferencia de 
    selection sort. Programe una rutina para implementar 
    insertion sort. ¿Qué tipo de complejidad tiene?

    El algoritmo \textit{insertion sort} recorre la lista de izquierda a derecha y,
    en cada paso, inserta el elemento actual en su posición correcta dentro de la 
    sublista ya ordenada situada a su izquierda. Así, después de cada inserción, 
    la parte izquierda de la lista permanece siempre ordenada.

    Por otro lado, \textit{selection sort} adopta en cada
    iteración busca el elemento mínimo del segmento aún desordenado y lo intercambia
    con el elemento ubicado al comienzo de dicho segmento. De este modo, va
    construyendo la lista ordenada seleccionando mínimos de manera sucesiva.

    \textbf{Algoritmo:}
    \verbatiminput{ordenamiento/ordenamiento/function/insertion_sort.txt }

    Este algoritmo tiene complejidad $O(n^2)$

    \item 
    Investigue uno de los siguientes ordenamientos rápidos: Quicksort, 
    Heapsort o Timsort. Programe una rutina para implementar el 
    ordenamiento seleccionado. ¿Qué tipo de complejidad tiene en el mejor 
    caso y peor caso?

    \textbf{Algoritmo:}
    \verbatiminput{ordenamiento/ordenamiento/function/quick_sort.txt }

    El mejor caso de este algoritmo es si el pivote divide excatamente a la
    mitad a las sucesivas listas, dando una complejidad de $O(n\,log(n))$. El
    peor caso, se da cuando el pivote es el menor o mayor elemento de la lista,
    dando una complejidad de $O(n^2)$.

    \item
    Realice el ordenamiento de la siguiente lista \([1,7,3,2,0,8]\) mostrando 
    paso a paso su ejecución mediante:
    \begin{enumerate}[label=\roman*.]
        \item Ordenamiento burbuja
          $$[1, 3, 7, 2, 4, 8]$$
          $$[1, 3, 2, 7, 4, 8]$$
          $$[1, 3, 2, 4, 7, 8]$$
          $$[1, 2, 3, 4, 7, 8]$$
        \item Ordenamiento por inserción
          $$[1, 7, 3, 2, 4, 8]$$
          $$[1, 3, 7, 2, 4, 8]$$
          $$[1, 2, 3, 7, 4, 8]$$
          $$[1, 2, 3, 4, 7, 8]$$
          $$[1, 2, 3, 4, 7, 8]$$
        \item Ordenamiento por selección
          $$[1, 7, 3, 2, 4, 8]$$
          $$[1, 2, 3, 7, 4, 8]$$
          $$[1, 2, 3, 7, 4, 8]$$
          $$[1, 2, 3, 4, 7, 8]$$
          $$[1, 2, 3, 4, 7, 8]$$
          $$[1, 2, 3, 4, 7, 8]$$
    \end{enumerate}

    \item 
    Realice el ordenamiento de la siguiente lista 
    \([22,36,6,79,26,45,75,13]\) mostrando paso a paso su ejecución mediante:
    \begin{enumerate}[label=\roman*.]
        \item Ordenamiento por fusión de listas (merge sort)
          $$[22, 36, 6, 79, 26, 45, 75, 13]$$
          $$[22, 36, 6, 79, 26, 45, 75, 13]$$
          $$[22, 36, 6, 79, 26, 45, 75, 13]$$
          $$[22, 36, 6, 79, 26, 45, 75, 13]$$
          $$[6, 22, 36, 79, 26, 45, 75, 13]$$
          $$[6, 22, 36, 79, 26, 45, 75, 13]$$
          $$[6, 22, 36, 79, 26, 45, 75, 13]$$
          $$[6, 22, 36, 79, 26, 45, 75, 13]$$
          $$[6, 22, 36, 79, 26, 45, 75, 13]$$
          $$[6, 22, 36, 79, 26, 45, 75, 13]$$
          $$[6, 22, 36, 79, 26, 45, 13, 75]$$
          $$[6, 22, 36, 79, 26, 45, 13, 75]$$
          $$[6, 22, 36, 79, 13, 26, 45, 75]$$
          $$[6, 22, 36, 79, 13, 26, 45, 75]$$
          $$[6, 22, 36, 79, 13, 26, 45, 75]$$
          $$[6, 22, 36, 79, 13, 26, 45, 75]$$
          $$[6, 22, 36, 79, 13, 26, 45, 75]$$
          $$[6, 13, 22, 36, 79, 26, 45, 75]$$
          $$[6, 13, 22, 36, 79, 26, 45, 75]$$
          $$[6, 13, 22, 26, 36, 79, 45, 75]$$
          $$[6, 13, 22, 26, 36, 79, 45, 75]$$
          $$[6, 13, 22, 26, 36, 45, 79, 75]$$
          $$[6, 13, 22, 26, 36, 45, 75, 79]$$
          $$[6, 13, 22, 26, 36, 45, 75, 79]$$
        \item Ordenamiento rápido (quick sort)
          $$[6, 36, 22, 79, 26, 45, 75, 13]$$
          $$[6, 13, 22, 79, 26, 45, 75, 36]$$
          $$[6, 13, 22, 26, 79, 45, 75, 36]$$
          $$[6, 13, 22, 26, 36, 45, 75, 79]$$
    \end{enumerate}

    Puede descargar estos ordenamientos de forma animada en \url{https://github.com/Corte0/programacion/tree/main/TP2/ordenamiento/mp4}
\end{enumerate}

